\documentclass[11pt]{article}

\usepackage{amssymb}
\usepackage{amsmath}
\usepackage{graphicx}
\usepackage{hyperref}

\def\N{{\mathbb N}}
\def\NN{{\mathcal N}}
\def\R{{\mathbb R}}
\def\E{{\mathbb E}}
\def\rank{{\mathrm{rank}}}
\def\tr{{\mathrm{trace}}}
\def\P{{\mathrm{Prob}}}
\def\sign{{\mathrm{sign}}}
\def\diag{{\mathrm{diag}}}

\setlength{\oddsidemargin}{0.25 in}
\setlength{\evensidemargin}{-0.25 in}
\setlength{\topmargin}{-0.6 in}
\setlength{\textwidth}{6.5 in}
\setlength{\textheight}{8.5 in}
\setlength{\headsep}{0.75 in}
\setlength{\parindent}{0.25 in}
\setlength{\parskip}{0.1 in}

\newcommand{\lecture}[4]{
   \pagestyle{myheadings}
   \thispagestyle{plain}
   \newpage
   \setcounter{page}{1}
   \setcounter{section}{0}
   \noindent
   \begin{center}
   \framebox{
      \vbox{\vspace{2mm}
    \hbox to 6.28in { {\bf A Mathematical Introduction to Data Science \hfill #4} }
       \vspace{6mm}
       \hbox to 6.28in { {\Large \hfill #1  \hfill} }
       \vspace{6mm}
       \hbox to 6.28in { {\it Instructor: #2\hfill #3} }
      \vspace{2mm}}
   }
   \end{center}
   \markboth{#1}{#1}
   \vspace*{4mm}
}


\begin{document}

\lecture{Homework 6. Cheeger Inequalities and Spectral Clustering}{Yuan Yao}{Due: Tuesday November 25, 2014}{December 2, 2014}

The problem below marked by $^*$ is optional with bonus credits. % For the experimental problem, include the source codes which are runnable under standard settings. 

\begin{enumerate}

\item {\em Spectral Bipartition}: Consider the 376-by-475 matrix $X$ of character-event for A Dream of Red Mansions, e.g. in the Matlab format 

\url{http://www.math.pku.edu.cn/teachers/yaoy/data/hongloumeng/hongloumeng376.mat} 

\noindent with a readme file:

\url{http://www.math.pku.edu.cn/teachers/yaoy/data/hongloumeng/readme.m}

Construct a weighted adjacency matrix for character-cooccurance network $A=X X^T$. Define the degree matrix $D=\diag(\sum_j A_{ij})$. Check if the graph is connected. 

\begin{enumerate}
\item Find the second smallest generalized eigenvector of $L=D-A$, i.e. $(D-A)f = \lambda_2 f$ where $\lambda_2>0$;
\item Sort the nodes (characters) according to the ascending order of $f$, such that $f_1\leq f_2 \leq \ldots \leq f_n$, and construct the subset $S_i = \{1,\ldots, i\}$;
\item Find an optimal subset $S^\ast$ such that the following is minimized  
\[ \alpha_f = \min_{S_i} \left\{ \frac{|\partial S_i|} {\min(|S_i|, |\bar{S}_i|)} \right\}\]  
where $|\partial S_i|=\sum_{x\sim y, x\in S_i, y\in \bar{S}_i} A_{xy}$ and $|S_i|=\sum_{x\in S_i} d_x = \sum_{x\in S_i, y} A_{xy}$.  
\item Check if $\lambda_2 > \alpha_f$;
\item Quite often people find a suboptimal cut by $S^+=\{i: f_i \geq 0\}$ and $S^-=\{i: f_i <0\}$. Compute its Cheeger ratio
\[ h_{S^+} =  \frac{|\partial S^+|} {\min(|S^+|, |S^-|)} \]
and compare it with $\alpha_f$, $\lambda_2$. 
\item You may further recursively bipartite the subgraphs into two groups, which gives a recursive spectral bipartition. 
\end{enumerate} 

\item {\em Directed Graph Laplacian}: Consider the following dataset with Chinese (mainland) University Weblink during 12/2001-1/2002,

\url{http://www.math.pku.edu.cn/teachers/yaoy/Fall2011/univ_cn.mat}

where $\tt{rank\_{}cn}$ is the research ranking of universities in that year, $\tt{univ\_{}cn}$ contains the webpages of universities, and $\tt{W\_{}cn}$ is the link matrix from university
$i$ to $j$. 

Define a PageRank Markov Chain
\[  P = \alpha P_0 + (1-\alpha) \frac{1}{n} e e^T, \ \ \ \alpha = 0.85 \]
where $P_0 = D_{out}^{-1} A$. Let $\phi\in \R_+^n$ be the stationary distribution of $P$, i.e. PageRank vector. Define $\Phi = \diag(\phi_i)\in \R^{n\times n}$.
\begin{enumerate}
\item Construct the normalized directed Laplacian 
\[ \mathcal{\vec{L}}=I - \frac{1}{2}(\Phi^{1/2} P \Phi^{-1/2} + \Phi^{-1/2} P^T \Phi^{1/2} ) \]
\item Use the second eigenvector of $\mathcal{\vec{L}}$ to bipartite the universities into two groups, and describe your algorithm in detail;
\item Try to explain your observation through directed graph Cheeger inequality.
\end{enumerate} 
  
%
%For your reference, an implementation of PageRank and HITs can be found at 
%
%\url{http://www.math.pku.edu.cn/teachers/yaoy/Fall2011/pagerank.m}


\item {\em *Chung's Short Proof of Cheeger's Inequality}: 

Chung's short proof is based on the fact that 
\begin{equation} 
h_G = \inf_{f\neq 0} \sup_{c\in \R} \frac{\sum_{x\sim y} |f(x) - f(y)|}{\sum_x |f(x) -c|d_x} 
\end{equation}
where the supreme over $c$ is reached at $c^*\in median(f(x):x\in V)$. Such a claim can be found in Theorem 2.9 in Chung's monograph, Spectral Graph Theory. In fact, Theorem 2.9 implies that the infimum above is reached at certain function $f$. From here, 
\begin{eqnarray}
\lambda_1 & =& R(f)=\sup_c \dfrac{\sum_{x\sim y}(f(x) - f(y))^2}{\sum_{x}(f(x)-c)^2 d_x},  \\
& \geq  &\dfrac{\sum_{x\sim y}(g(x) - g(y))^2}{\sum_{x}g(x)^2 d_x}, \ \ \ g(x)=f(x)-c\\
& = & \dfrac{(\sum_{x\sim y}(g(x) - g(y))^2)(\sum_{x\sim y}(g(x) +g(y))^2)}{(\sum_{x\in V}g^2(x)d_x)((\sum_{x\sim y}(g(x) +g(y))^2)} \\
& \ge & \dfrac{(\sum_{x\sim y}|g^2(x) - g^2(y)|)^2}{(\sum_{x\in V}g^2(x)d_x)((\sum_{x\sim y}(g(x) +g(y))^2)} , \ \ \textrm{Cauchy-Schwartz Inequality} \\
& \ge & \dfrac{(\sum_{x\sim y}|g^2(x) - g^2(y)|)^2}{2( \sum_{x\in V}g^2(x)d_x)^2} , \ \ \textrm{$(g(x)+g(y))^2\leq 2 (g^2(x)+g^2(y))$} \\
& \ge & \dfrac{h_G^2}{2}.
\end{eqnarray}
Is there any step wrong in the reasoning above? If yes, can you remedy it/them?

\end{enumerate}

\end{document}


