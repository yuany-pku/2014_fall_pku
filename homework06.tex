\documentclass[11pt]{article}

\usepackage{amssymb}
\usepackage{amsmath}
\usepackage{graphicx}
\usepackage{hyperref}

\def\N{{\mathbb N}}
\def\NN{{\mathcal N}}
\def\R{{\mathbb R}}
\def\E{{\mathbb E}}
\def\rank{{\mathrm{rank}}}
\def\tr{{\mathrm{trace}}}
\def\P{{\mathrm{Prob}}}
\def\sign{{\mathrm{sign}}}
\def\diag{{\mathrm{diag}}}

\setlength{\oddsidemargin}{0.25 in}
\setlength{\evensidemargin}{-0.25 in}
\setlength{\topmargin}{-0.6 in}
\setlength{\textwidth}{6.5 in}
\setlength{\textheight}{8.5 in}
\setlength{\headsep}{0.75 in}
\setlength{\parindent}{0.25 in}
\setlength{\parskip}{0.1 in}

\newcommand{\lecture}[4]{
   \pagestyle{myheadings}
   \thispagestyle{plain}
   \newpage
   \setcounter{page}{1}
   \setcounter{section}{0}
   \noindent
   \begin{center}
   \framebox{
      \vbox{\vspace{2mm}
    \hbox to 6.28in { {\bf A Mathematical Introduction to Data Science \hfill #4} }
       \vspace{6mm}
       \hbox to 6.28in { {\Large \hfill #1  \hfill} }
       \vspace{6mm}
       \hbox to 6.28in { {\it Instructor: #2\hfill #3} }
      \vspace{2mm}}
   }
   \end{center}
   \markboth{#1}{#1}
   \vspace*{4mm}
}


\begin{document}

\lecture{Homework 5. Perron-Frobenius and Fiedler Theory}{Yuan Yao}{Due: Tuesday November 18, 2014}{November 25, 2014}

The problem below marked by $^*$ is optional with bonus credits. % For the experimental problem, include the source codes which are runnable under standard settings. 

\begin{enumerate}

\item {\em PageRank}:  The following dataset contains Chinese (mainland) University Weblink during 12/2001-1/2002,

\url{http://www.math.pku.edu.cn/teachers/yaoy/Fall2011/univ_cn.mat}

where $\tt{rank\_{}cn}$ is the research ranking of universities in that year, $\tt{univ\_{}cn}$ contains the webpages of universities, and $\tt{W\_{}cn}$ is the link matrix from university
$i$ to $j$. 
 
\begin{enumerate}
\item Compute PageRank with Google's hyperparameter $\alpha=0.85$;
\item Compute HITS authority and hub ranking using SVD of the link matrix; 
\item Compare these rankings against the research ranking (you may consider Kendall's $\tau$ distance -- as the number of pairwise mismatches between two orders -- to compare different rankings);  
\item Compute extended PageRank with various hyperparameters $\alpha\in (0,1)$, investigate its effect on ranking stability. 
\end{enumerate} 

For your reference, an implementation of PageRank and HITs can be found at 

\url{http://www.math.pku.edu.cn/teachers/yaoy/Fall2011/pagerank.m}

\item {\em Perron Theorem:} Assume that $A>0$. Consider the following optimization problem:
\begin{eqnarray*}
& & \max \delta \\
& s.t. & Ax\geq\delta x \\
&& x\geq 0 \\
&& x \neq 0.
\end{eqnarray*}
Let $\lambda^*$ be optimal value with $\nu^*\geq0,\quad 1^T\nu^*=1$, and $A\nu^*\geq\lambda^*\nu^*$. Show that
\begin{enumerate}
\item $A\nu^*=\lambda^*\nu^*$, i.e. $(\lambda^*,\nu^*)$ is an eigenvalue-eigenvector pair of $A$;\\
\item  $\nu^*>0$;\\
\item[*(c)]$\lambda^*$ is unique and $\nu^*$ is unique;\\
\item[*(d)]  For other eigenvalue $\lambda\quad(\lambda z=Az\quad when\quad z\neq0)$, $|\lambda|<\lambda^*$.
\end{enumerate}


\item {\em Absorbing Markov Chain:} 

Let $P$ be a row Markov matrix on $n+1$ states with non-absorbing state $\{1,\ldots,n\}$ and absorbing state $n+1$. Then $P$ can be partitioned into 
\[
	P=\left[
	\begin{array}{cc} 
	Q & R \\ 
	0 & 1 
	\end{array} \right]
\]
Assume that $Q$ is primitive. Let $N(i,j)$ be the expected number of jumps starting from nonabsorbent state $i$ and hitting state $j$, before reaching the absorbing state $n+1$. Show
that
\begin{enumerate}
\item $N(i,i) = 1 + \sum_k N(i,k) Q(k,i)$, for $i=1,\ldots,n$;
\item $N(i,j) = \sum_k N(i,k) Q(k,j)$, for $i\neq j$;
\item These identities together imply that $N=(I-Q)^{-1}$, called the fundamental matrix; 
\item Show that the probability of absorption from state $i$, $B(i)$ ($i=1\ldots,n$), is given by $B=NR$.   
\end{enumerate}


\end{enumerate}

\end{document}


